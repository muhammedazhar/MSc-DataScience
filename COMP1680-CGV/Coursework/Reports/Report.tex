\documentclass{article}
\usepackage{microtype}
\usepackage{graphicx}
\usepackage{subfigure}
\usepackage{subcaption}
\usepackage{booktabs}
\usepackage[style=authoryear,backend=biber]{biblatex}
\addbibresource{References.bib}
\usepackage{amsmath}
\usepackage{amsfonts}
\usepackage{amsthm}
\usepackage{physics}
\usepackage{fancyvrb}
\usepackage{listings}
\usepackage{xcolor}
\usepackage{caption}
\usepackage{xurl}
\usepackage{hyperref}
\newcommand{\theHalgorithm}{\arabic{algorithm}}
\usepackage{accessibility}
\usepackage{Coursework}

\begin{document}
\cwtitle{COMP1680 - Clouds, Grids and Virtualisation Coursework Report}
\begin{cwauthorlist}
\cwauthor{Azhar Muhammed - 001364857}
\cwauthor{Word Count: }
\end{cwauthorlist}

\section{Executive Summary}

\section{Part 1: Parallel Processing using Cloud Computing}

\subsection{Analysis}

\subsection{Comparison}

\subsection{Recommendation}

\section{Part 2: Parallel Programming Exercise}

\subsection{Step 1}

Here's an example of the Jacobi code with GitHub-style formatting:

\begin{lstlisting}[style=CStyle, caption={Example of Jacobi 2D Implementation}]
#include <stdio.h>
#include <stdlib.h>
#include <math.h>
#include <sys/time.h>

#define TOP_TEMP    15.0
#define BOTTOM_TEMP 60.0
#define LEFT_TEMP   47.0
#define RIGHT_TEMP  100.0

void initialize_grid(double **grid, int n, int m) {
    int i, j;
    
    // Initialize interior points to 0
    for (i = 1; i < n-1; i++) {
        for (j = 1; j < m-1; j++) {
            grid[i][j] = 0.0;
        }
    }
    
    // Set boundary conditions
    for (i = 0; i < n; i++) {
        grid[i][0] = LEFT_TEMP;      // Left boundary
        grid[i][m-1] = RIGHT_TEMP;   // Right boundary
    }
    for (j = 0; j < m; j++) {
        grid[0][j] = TOP_TEMP;       // Top boundary
        grid[n-1][j] = BOTTOM_TEMP;  // Bottom boundary
    }
}
\end{lstlisting}

\subsection{Step 2}

\subsection{Step 3}

\section{Conclusions}

% \printbibliography

\end{document}