%%%%%%%% COMP1801 Coursework Template File %%%%%%%%%%%%%%%%%
% This template and document is based on \href{https://media.icml.cc/Conferences/ICML2021/Styles/icml2021\_style.zip}{ICML 2021 LaTeX style file} (\url{https://media.icml.cc/Conferences/ICML2021/Styles/icml2021\_style.zip})
% Modified from a template file originally constructed by Atsushi Suzuki and Jing Wang
% Copyright (the body text): Peter Soar, 2022-2024.

%%%%%%%%%%%%%%%%%%%%%%%%%%%%%%%%%%%%%%%%%%%%%%%%%%%%%%%%%%%%%%%%%%%%%

% This document is the template for the COMP1801 Machine learning coursework. Please make a copy to your own overleaf account so you can start editing it.

%Please do not change any part of the layout (font, margins, extra sections (you can add subsections if you like) or anything else that involves editing the .sty files

% I do not think you should need to include any packages other than the ones I have included already, but you may do so if you feel it necessary.

%%%%%%%%%%%%%%%%%%%%%%%%%%%%%%%%%%%%%%%%%%%%%%%%%%%%%%%%%%%%%%%%%%%%%%%%%
% Header Information and Packages
% Scroll down to the main document
%%%%%%%%%%%%%%%%%%%%%%%%%%%%%%%%%%%%%%%%%%%%%%%%%%%%%%%%%%%%%%%%%%%%%%%%%

\documentclass{article}
% Recommended, but optional, packages for figures and better typesetting:
\usepackage{microtype}
\usepackage{graphicx}
\usepackage{subfigure}
\usepackage{subcaption}
\usepackage{booktabs} % for professional tables

% Uncomment the following two lines if you intend to use references
\usepackage[style=authoryear,backend=biber]{biblatex}
\addbibresource{ref.bib}% Syntax for version >= 1.2

\usepackage{amsmath}
\usepackage{amsfonts}
\usepackage{amsthm}
\usepackage{physics}
\usepackage{fancyvrb}
\usepackage{caption}
% hyperref makes hyperlinks in the resulting PDF.
\usepackage{xurl}
\usepackage{hyperref}

% Attempt to make hyperref and algorithmic work together better:
\newcommand{\theHalgorithm}{\arabic{algorithm}}

\usepackage{accessibility}

\usepackage{Coursework}

\begin{document}
%%%%%%%%%%%%%%%%%%%%%%%%%%%%%%%%%%%%%%%%%%%%%%%%%%%%%%%%%%%%%%%%%%%%%%%%%
% Main document begins
% From here onward contains the actual document contents that will be seen in the compiled PDF
%%%%%%%%%%%%%%%%%%%%%%%%%%%%%%%%%%%%%%%%%%%%%%%%%%%%%%%%%%%%%%%%%%%%%%%%%
% This is the Coursework title - Do not change
\cwtitle{COMP1801 - Machine Learning Coursework Report}


\begin{cwauthorlist}
% Please provide your details on this line
\cwauthor{Azhar Muhammed - 001364857}

% Please provide your final word count
\cwauthor{Word Count:}
\end{cwauthorlist}

% Below will be the main body of your document. Please do not change the headings for the sections. You may add subsections if you feel it appropriate, but this is not required.

% If you are struggling with Latex and do not know how to include figures, tables and/or equations, please see my example document provided in week 1 on moodle and at: https://www.overleaf.com/project/6330ba64217dc3332f06fb4c
% Also other resources such as: https://v1.overleaf.com/latex/templates/a-quick-guide-to-latex/fghqpfgnxggz.pdf

% I have included the brief for each section of the report as comments, feel free to delete these comments in your final version
% If there is any inconsistency between the instructions in these comments and the instructions in the Coursework Specification pdf, then the CW specification pdf should **always** be assumed as being the correct document, but if there is ever any doubt please contact me at p.soar@gre.ac.uk.

\section{Executive Summary}

% This should be a summary of what the report contains – the problem you are solving, why it's important, the ML methods you have used and a summary of your results and conclusions. Avoid generic statements about machine learning, A.I., data processing etc., and keep to the specifics of the coursework.

The purpose of this report is to develop machine learning models to predict the lifespan of metal parts and determine whether they are defective based on a given dataset. The project addresses two key areas: regression to predict lifespan and classification to determine defect status. Regression was implemented using models like Linear Regression, Random Forest, and an Artificial Neural Network (ANN), with ANN ultimately chosen for its robustness and ability to capture complex relationships. The classification task involved Logistic Regression and ANN to label parts as defective or not based on a lifespan threshold of $1500$ hours. Through careful feature crafting, preprocessing, and evaluation using metrics such as $\text{RMSE}$, $\text{R}^2$, Weighted $\text{F1}$-Score, and Recall, the study concludes that the ANN model is preferable for both regression and classification tasks due to its superior performance in handling complex patterns, class imbalance, and providing accurate predictions.

\section{Data Exploration}
%Describe how you loaded the dataset, then perform a concise exploration of the data and comment on any patterns or relationships in the data that you think may be relevant for creating a model and predicting the part lifespan. It is advisable to provide plots and visualizations to better highlight these regions of interest. 

%Considering this data exploration, identify the features you will use in your models and discuss why you believe they will be the best predictors of metal part longevity.

%Finally, with reference to patterns/relationships found in the exploration and (if relevant) theoretical justification, provide a brief additional discussion outlining your expectations of which approach (i.e. regression or classification) and specific ML model will be most appropriate for providing an accurate solution to the company's requirements.

The purpose of this data exploration is to understand the dataset, identify significant relationships among features, and determine the best predictors for modeling metal part lifespan. By employing visualizations such as scatter plots, regression lines, and polynomial feature transformations, this project aims to elucidate the factors influencing part longevity and guide model development to ensure robust, precise, and interpretable predictive models.

The dataset was loaded using Python's \texttt{pandas} library, and inconsistencies or missing values were addressed. The target variable, \texttt{Lifespan}, had $998$ unique values out of $1000$ data points, indicating an almost entirely unique dataset.

Initial exploratory data analysis (EDA) was conducted using scatter plots, histograms, and other visualizations to understand relationships between features and \texttt{Lifespan} and to identify linear and non-linear trends.

The analysis revealed that \texttt{coolingRate} had a very weak positive correlation with \texttt{Lifespan}, leading to its tentative inclusion in the model. `quenchTime` and `forgeTime` showed slight positive correlations, suggesting that extending these processes could enhance metal durability by improving internal structure.

Alloy composition analysis showed that \textbf{Nickel\%} had a weak positive correlation, likely enhancing lifespan through corrosion resistance, while \textbf{Iron\%} showed a slight negative correlation, possibly reducing durability due to brittleness. \textbf{Cobalt\%} and \textbf{Chromium\%} exhibited weak positive trends, consistent with their role in improving metal strength.

\textbf{smallDefects} required a polynomial transformation to capture its parabolic relationship with \texttt{Lifespan}. Initially, small defects had little impact, but beyond a threshold, they reduced durability. Based on these observations, \texttt{coolingRate} and \texttt{smallDefects} (with a quadratic term) were selected as key features for modeling lifespan. All features were thoroughly considered, as the intention is to implement a Neural Network in future tasks to capture all potential interactions and non-linearities.

The high number of unique \texttt{Lifespan} values ($998$ out of $1000$) favored a regression approach for predicting continuous outcomes. Polynomial Regression was applied to capture non-linear relationships for \texttt{coolingRate} and \texttt{smallDefects}.

In summary, the data exploration provided valuable insights into feature relationships with \texttt{Lifespan}, informing both feature selection and modeling approaches for robust durability prediction.
\section{Regression Implementation}
This section focuses on regression as a method to predict the lifespan of metal parts using the given dataset. Regression analysis was used to develop a model capable of accurately predicting the lifespan based on features such as alloy composition, manufacturing conditions, and defects. In this context, an Artificial Neural Network (ANN) was chosen due to its robustness and capability to consider all features for predicting lifespan, whereas the Random Forest and XGBoost models focused specifically on the most important features. The following sections will detail the chosen regression models, the preprocessing and tuning methodologies, and an evaluation of their performances, aiming to establish the most suitable regression model for this task.
\subsection{Methodology}
%The task for this section is to create a regression model methodology to predict the lifetime of a metal part using machine learning methods.

%Firstly, choose two model types appropriate for this task, for example Linear Regression and Artificial Neural Network. Both models chosen should be supported with a brief justification explaining why it is an appropriate choice for the problem.

%Secondly, describe and implement an appropriate pre-processing routine to be used in all following experimentations and evaluations of this section only (Part 3). This can include, but is not limited to: categorical feature encoding, feature scaling and splitting the data into training and test sets, data balancing, etc.

%Thirdly, provide and justify the hyper-parameter tuning framework you will use to obtain your final models in the experiments in section 3.2. Describe which hyper-parameters you will tune for each model type chosen, how they work and why they are essential to the optimization of that architecture.

Two regression models were chosen for this task: Linear Regression and Artificial Neural Network. Linear Regression was selected for its simplicity and as a baseline to compare against more complex methods, whereas Random Forest Regressor was selected due to its ability to handle non-linear relationships and interactions without requiring extensive feature engineering. During initial testing, tree-based models, particularly Random Forest and XGBoost, demonstrated superior performance compared to other models, such as Artificial Neural Networks (ANNs). However, in the Random Forest and XGBoost models where only focusing on high importance features. I have chose ANN due its robustness and considering all features for predicting lifespan.
The dataset contained mixed data types, necessitating careful preprocessing to standardize and encode features appropriately. StandardScaler was applied to normalize numerical features, which improved model convergence and stability by scaling features to have a mean of 0 and standard deviation of 1. This choice of scaler was particularly suited due to the approximate normal distribution of several numerical features.

To address categorical variables, a combination of \textbf{One-Hot Encoding}, \textbf{Label Encoding} and a hybrid approach was employed. The hybrid approach ensured that high-cardinality categorical variables  were appropriately represented, while the partType column were treated with Label Encoding to retain ordinal relationships where applicable. However, this approach shown low results while One-Hot encoding performed better. A consistent train-test split of 70\% training, 15\% validation, and 15\% testing was used across all models to ensure fair comparison. The validation set was utilized for hyperparameter tuning, while the test set remained untouched to provide an unbiased evaluation.
To avoid overfitting and ensure that the model generalizes well, **cross-validation** was implemented along with a validation set. This allowed better use of the available data for both training and validation, thus enhancing the reliability of the model performance estimates.
Hyperparameter tuning was performed for each model to optimize performance. For Linear Regression, regularization techniques such as Lasso and Ridge regression were considered to improve model stability and prevent overfitting. For the Artificial Neural Network, key hyperparameters including learning rate, number of units per layer, and dropout rates were tuned using RandomSearchCV. After conducting 150 trials, the Sequential Neural Network achieved an optimal validation loss of 12,098.92. The best architecture consisted of two layers, with the first layer containing 64 units and a dropout rate of 0.1, and the second layer containing 112 units with a dropout rate of 0.2. An optimal learning rate of 0.00566 was identified. The evaluation metrics for the Neural Network model on the test set, using a wrapper class, were as follows: RMSE = 113.29, R² = 0.90, and MAE = 90.19. This relatively simple architecture, combined with batch normalization and dropout, effectively balanced model complexity and performance. While a final validation loss of 61,680.13 was recorded in one trial, the configuration with the best observed performance can serve as a strong baseline for future work. The consistent use of batch normalization and moderate dropout rates provided effective regularization, enhancing the model's ability to capture complex interactions without overfitting.
\subsection{Evaluation}
%For this sub-section you should perform and describe the experiments done to obtain your final regression model version for each chosen architecture type, by adhering to the hyper-parameter tuning framework outlined above. These experiments should be a rigorous model optimization process including comparisons between the different versions via a table or otherwise detailed description. All choices should be justified using theory, experiments and/or references. 
%The final model versions of both chosen types should be described in all detail – summarizing all relevant final hyperparameters chosen.
%You should evaluate these final model versions using a test portion of the dataset not used in the prior training stages, using appropriate regression performance metrics. You should explain how these chosen metrics work, why they are appropriate to the task and provide a written interpretation of how well your model is performing at the given task according to these metrics. 
%Finally, compare the best performing version of both model types using your chosen performance metrics. Provide your final recommendation of which model is superior to deploy for this regression task, supporting this choice with a brief discussion of your results.  
The chosen regression models were trained on the training set and evaluated using the testing set to compare their effectiveness in predicting part lifespan. Each model's performance was quantified using several metrics such as R2, RMSE and MAE that are particularly well-suited for regression tasks involving continuous numerical outputs.\subsection{Critical Review}
%For this sub-section critically review your overall methodology used for this task only (Part 3), considering the results obtained in your experiments in section 3.2. Cover areas of strengths and areas where improvement might be needed. Offer an alternative approach from the choices not utilized in your experimentations that future investigations could explore. These can include untrialled model architectures, alternate pre-processing routines, different hyper-parameter tuning schemes, etc.  
The metrics used to evaluate the models included Root Mean Squared Error (RMSE), Mean Absolute Error (MAE), and R² Score. RMSE was selected for its ability to penalize larger errors, which is crucial when predicting lifespans that may vary significantly. MAE provided a straightforward measure of the average error, and R² Score quantified the proportion of variance explained by the model. These metrics offered comprehensive insights into both prediction accuracy and error magnitude.
The initial comparison of the models highlighted that the Random Forest Regressor outperformed Linear Regression in terms of RMSE and R² Score, capturing the non-linear relationships in the data more effectively. While Linear Regression provided a simple baseline, it struggled with the complex dependencies inherent in the dataset. In contrast, the Random Forest Regressor demonstrated superior flexibility and performance.
Despite the strong performance of Random Forest, a finely tuned **Artificial Neural Network (ANN)** was ultimately chosen due to its robustness in considering all features and its ability to model complex relationships comprehensively. The ANN achieved an RMSE of 113.29, an R² of 0.90, and an MAE of 90.19, indicating a high level of accuracy in lifespan prediction. In comparison, Random Forest and XGBoost models focused primarily on high-importance features but did not achieve the same level of accuracy across all metrics.

The final step involved integrating all processes into a single **pipeline**, which combined preprocessing, model training, and hyperparameter tuning. This approach ensured that the resulting model was both efficient to deploy and consistent in performance across various scenarios.
\section{Classification Implementation}
\subsection{Feature Crafting}
%Your line manager has decided that beyond the exact lifetime of a part, it is also important to know whether a part has a lifetime above 1500 hours (determined by the company as the minimum lifetime before a part is considered defective).

%Before choosing any potential model architectures, first create an additional feature (a column named "1500_labels" for example) representing a binary output label for this lifespan threshold that may be used to predict whether a part is defective or not. Populate this feature with a positive binary output if the hourly threshold is met. 
%However, your line manager thinks that splitting the data into only two groups may be naïve, and hence predictions made by a binary classification model may not be suitable for finding the best processing parameters for manufacture. Currently, it is unclear how many groups the data should be split into and why.
%nstead of using the provided threshold of 1500 to create a binary class, you can (for higher marks) perform and utilize alternate grouping methods on the records based on the lifespan and potentially relationships with the other features, while providing justifications for doing so. The type of this grouping can range from a more complex thresholding technique (such as a statistical outlier function) to the application of an unsupervised learning algorithm, such as clustering, to separate the inputs into different groups (which can result in three or more groups.) More complex and/or better-justified methods will earn higher marks for this section than simpler ones, but must be researched by the student on their own. 

\subsection{Methodology}
%The task for this section is to create a classification model methodology to predict the output class of a metal part using machine learning methods.
%Firstly, once your output labels are created, choose two model types appropriate for this task, for example Logistic Regression and Artificial Neural Network. Both models chosen should be supported with a brief justification explaining why it is an appropriate choice for the problem.
%Secondly, describe and implement an appropriate pre-processing routine to be used in all following experimentations and evaluations of this section only (Part 4). This can include, but is not limited to: categorical feature encoding, feature scaling and splitting the data into training and test sets, data balancing, etc.
%Thirdly, provide and justify the hyper-parameter tuning framework you will use to obtain your final models in the experiments in section 4.3. Describe which hyper-parameters you will tune for each model type chosen, how they work and why they are essential to the optimization of that architecture.   

\subsection{Evaluation}
%For this sub-section you should perform and describe the experiments done to obtain your final classification model version for each chosen model type, by adhering to the hyper-parameter tuning framework outlined above. These experiments should be a rigorous model optimization process including comparisons between the different versions via a table or an otherwise detailed description. All choices should be justified using theory, experiments and/or references. 
%The final model versions of both chosen types should be described in all detail – summarizing all relevant final hyperparameters chosen.
%You should evaluate your final model versions using a test portion of the dataset not used in the prior training stages, using appropriate classification performance metrics. You should explain how these chosen metrics work, why they are appropriate to the task and provide a written interpretation of how well your model is performing at the given task according to these metrics. 
%Finally, compare the best performing version of both model types using your chosen performance metrics. Provide your final recommendation of which model is superior to deploy for this classification task, supporting this choice with a brief discussion of your results.

\subsection{Critical Review}
%For this sub-section critically review your overall methodology used for this task only (Part 4) considering the results obtained in your experiments. Cover areas of strengths and areas where improvement might be needed. Offer an alternative approach for future investigations to explore from the choices not utilized in your experimentations. These can include untrialled model architectures, alternate pre-processing routines, different hyper-parameter tuning schemes, etc.  


\section{Conclusions}
% Provide a final recommendation to your manager for which model to use between the Regression (Part 3) or Binary Classification (Part 4) implementation for the task of predicting the lifetime of metal parts, providing a reasoned justification for this recommendation considering aspects such as (but not limited to) model accuracy, the clustering results, and the outside context of the task. 
% Considering the final CNN model chosen for classifying the defects of metal parts (Part 5), you should briefly discuss if you think the model is good enough to be deployed and used in practice (justifying why) and provide your top suggestion to improve it.


% The command below will create a reference list for you. 
\printbibliography

\end{document}